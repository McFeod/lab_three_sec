Допустим, Виктор должен удостовериться, что отправителем сообщения $T$ является Пегги.

\begin{enumerate}

\item Пегги вычисляет хеш исходного сообщения $h = h(T)$ (512 бит);
\item Пегги вычисляет $e = h \ mod \ q$;
\item Пегги выбирает $k$, меньшее $q$ из открытого ключа;
\item Пегги вычисляет $C(x_C, y_C) = k \cdot P(x_P, y_P)$;
\item Пегги вычисляет $r = x_C \ mod \ q$;
\item Если $r = 0$, выбирается другое $k$;
\item Пегги вычисляет $s = (r \cdot d + k \cdot e) \ mod \ q$, используя закрытый ключ $d$;
\item Если $s = 0$, выбирается другое $k$;
\item Пегги отправляет Виктору $T$ и подпись $\left\{r, s \right\}$;

\end{enumerate}
